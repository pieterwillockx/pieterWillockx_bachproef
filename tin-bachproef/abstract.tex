\begin{abstract}
% TODO: De "abstract" of samenvatting is een kernachtige (max 1 blz. voor een
% thesis) synthese van het document. In ons geval beschrijf je kort de
% probleemstelling en de context, de onderzoeksvragen, de aanpak en de
% resultaten.
% \lipsum[1-4]
De informatica-business staat niet stil. Dat weet men van het moment dat men er de eerste stappen in zet. Een welbekende quote uit de IT-wereld, \emph{The one certainty is change.}, is hier het levende bewijs van. Een ontwikkelaar moet daarom vrede nemen met het feit dat hij in deze branche levenslang zal moeten bijleren.
\\
De IT-business evolueert constant. Gebruikers passen zich aan aan nieuwe systemen die op de markt komen en verwachten van de ontwikkelaar dat hij zich hier ook aan aanpast, en de tijdlijn van applicatieontwikkeling is hier het beste voorbeeld van.
\\
Vroeger was het aanleveren van oplossingen aan de klant een gemakkelijkere zaak. Applicaties werden geschreven in één enkele codetaal en konden op één soort client draaien, in bijna elk geval een desktop-machine. De gehele applicatie werd gebundeld in één consistent pakket: Een monolithische applicatie die alle logica en functionaliteit bevatte. De dag van vandaag hebben we echter te maken met een ontwikkeling in de business waar dit soort architectuur niet tegen opgewassen is. 
\\
Tegenwoordig bestaat er een groot aantal verschillende soorten clients die elk op een zeer specifieke manier met data omgaan. Aan de ene kant heb je desktop-clients die beschikken over relatief grote processoren en een krachtige internetverbinding. Langs de andere kan zijn er de mobiele clients zoals smartphones, tablets, smartwatches en degelijke. Deze soorten toestellen zijn ontwikkeld om draagbaar te zijn, en beschikken dus over minder rekenkracht en een internetverbinding die soms instabiel kan zijn, afhankelijk van de locatie waar de gebruiker zich bevind. 
\\
Hoewel de gebruikers nu verspreid zijn over verschillende soorten clients wil men toch nog steeds een applicatie kunnen ontwikkelen die door zoveel mogelijk van deze gebruikers kan worden geconsumeerd.
In dit scenario is de monolithische manier van ontwikkelen verouderd. Er is nood aan een nieuwe architectuur die toestaat om één applicatie te schrijven die voor elk soort toestel dezelfde kwaliteit van gebruik kan bieden.
\\
In dit schrijven wordt een architectuur geanalyseerd die een oplossing biedt voor deze problemen. Een architectuur die in de laatste jaren enorm aan populariteit heeft gewonnen en wordt toegepast in enkele van de grootste bedrijven die vandaag competitief zijn op de markt. De "Microservices Architecture".
De grootste voordelen en nadelen worden besproken en vergeleken met de monolithische architectuur, en er wordt getracht om deze stellingen te bewijzen aan de hand van simulaties waarbij parameters zoals latency, geheugengebruik en andere gemeten en vergeleken worden. Verder wordt er ook gekeken naar de mogelijke barrières die momenteel bestaan voor bedrijven die het moeilijk maken om de overstap naar deze nieuwere architectuur te maken.
\end{abstract}